\documentclass{article}    
\usepackage[letterpaper,margin=1in]{geometry}
\usepackage{lastpage}
\usepackage{graphicx}
\usepackage{fancyhdr}
\usepackage{enumitem}
\usepackage{amssymb}
\usepackage{amsmath}

\def\ojoin{\setbox0=\hbox{$\bowtie$}%
  \rule[-.02ex]{.25em}{.4pt}\llap{\rule[\ht0]{.25em}{.4pt}}}
\def\leftouterjoin{\mathbin{\ojoin\mkern-5.8mu\bowtie}}
\def\rightouterjoin{\mathbin{\bowtie\mkern-5.8mu\ojoin}}
\def\fullouterjoin{\mathbin{\ojoin\mkern-5.8mu\bowtie\mkern-5.8mu\ojoin}}

\fancypagestyle{plain}{%
\fancyhf{}%
\lhead{ECS165A SQ18}
\rhead{\today}
\cfoot{Homework 4\\ \thepage\ of \pageref{LastPage} }
\renewcommand{\headrulewidth}{0.4pt}
\renewcommand{\footrulewidth}{0.4pt}
}

\pagestyle{fancy}
\fancyhf{}
\lhead{ECS165A SQ18}
\rhead{\today}
\cfoot{Homework 4\\ \thepage\ of \pageref{LastPage} }
\renewcommand{\headrulewidth}{0.4pt}
\renewcommand{\footrulewidth}{0.4pt}

\begin{document}

\title{Homework 4}
\author{Name: Brandon Tam\\SID: 913892839\\Partner Name: Shivang Soni\\Partner SID: 915623718}

\maketitle

Due 11:59PM June 5, 2018. {\bf READ ALL DIRECTIONS VERY CAREFULLY!} 
Submit your code, tex files along with a generated PDF. {\bf DO NOT SUBMIT DATA FILES!} 
For this homework you will be working in groups of two, a group of three will only be allowed with approval due to odd number of students. 
All programs will be evaluated on the CSIF. Upload your files as a tar gzip file (tgz). Only submit one homework per partner. This specification is subject to change.

You are designing a database for a university called FakeU. As a trial you have been provided grade data from courses for departments ABC and DEF. 
The grade data is from Summer of 1989 until Summer of 2012. The data provided is in CSV format, and is only as complete as could be made possible. 
There may be errors, omissions or redundant data in the files. 
FakeU like UC Davis is on a quarter system, however they have recently transitioned to a single summer quarter instead of two summer sessions. 
This has corrupted some of their summer data as all summer session classes have now been grouped into a single summer quarter term. 
Each course has a course ID (CID), a term it was offered (TERM), a subject (SUBJ), a course number (CRSE), a section (SEC), and number of units (UNITS). 
Within a course there listings of meetings, the instructor of the meeting (INSTRUCTOR(S)), meeting type (TYPE), day of meeting (DAYS), time of meeting (TIME), meeting building (BUILD), and meeting room (ROOM) are also listed. For each student that takes the course there is a student seat (SEAT), a student ID (SID), the student�s surname (SURNAME), the student�s preferred name (PREFNAME), the student�s (LEVEL), the number of units the student is receiving (UNITS), the student�s class standing (CLASS), the student�s major (MAJOR), the grade the student received in the course (GRADE), the student�s registration status (STATUS), and the student�s e-mail address (EMAIL).
There may be courses that are cross listed between the two departments (e.g. ABC 123 may be cross listed as DEF 456).

You {\bf MUST} put each problem on a separate page with 1a on the second page, for example 1a will be on page 2 and 1b will be on page 3 (this template is already setup for this). 
You {\bf MUST} put your name and student ID in the provided author section above. {\bf FAILURE TO DO SO MAY RESULT IN NO CREDIT!} 
The data will be provided on Canvas, and the CSV files will also be on the CSIF in /home/cjnitta/ecs165a/Grades. All submissions will be compared with MOSS, including against past submissions.

Some useful tips: 
\begin{itemize}
\item When loading the tuples into the database, insert them in batches. Inserting one tuple at a time may cause the program to take on the order of tens of minutes or hours instead of a few minutes.
\item Test a subset of the data first.
\end{itemize}

\clearpage
\section*{Part 1}
You will be creating a database schema for your grade data.

\begin{enumerate}[label=\alph*.]
\item Provide an ER diagram for your database schema. Only include images generated from vector based programs. \\
% Problem 1a goes below

$\{B\}^+ = \{ B,C\}$ \\
$\{A\}^+ = \{ A\}$ \\
$\{C\}^+ = \{ C\}$ \\
$\{D\}^+ = \{ D\}$ \\
$\{E\}^+ = \{ E\}$ \\
$\{AB\}^+ = \{A,B,C\}$ \\
$\{AC\}^+ = \{ A,C\}$ \\
$\{AD\}^+ = \{A,D\}$ \\
$\{AE\}^+ = \{ A,E\}$ \\
$\{BC\}^+ = \{B,D\}$ \\
Key:$\{BD\}^+ = \{ B,C,D,E,A\}$ \\
Key:$\{BE\}^+ = \{B,C,D,E,A\}$ \\
$\{CD\}^+ = \{ C,D,E,A\}$ \\
$\{CE\}^+ = \{ C,E\}$ \\
$\{DE\}^+ = \{ D,E,A\}$ \\

Hence from the above closure the Keys of the relation R is $\{B,D\}$ and $\{B,E\}$ \clearpage
\item Provide a description of the tables in your schema, and their attributes. Make sure you describe how you will store the instructor, student, building, course, etc. information. \\
% Problem 1b goes below
. \\ \\ \\
$\{B\}^+ = \{ B\}$ \\
$\{B,E\}^+ = \{ C,B,E\}$ \\
$\{D,C\}^+ = \{ D,C\}$ \\
$\{D,E\}^+ = \{ D,E\}$ \\ \\

Try to remove the attributes from the functional dependencies and make sure if the RHS attribute can be reached. \\ \\ \\

From the relation $BE -> D$\\
Try removing E$\{B\}^+ = \{B,C\}$ \\
Try removing B$\{E\}^+ = \{ E\}$ \\ \\ \\
Similarly, From the relation $DC -> E$\\
Try removing D$\{C\}^+ = \{C\}$ \\
Try removing C$\{D\}^+ = \{ D\}$ \\ \\ \\

and also From the relation $DE -> A$\\
Try removing D$\{E\}^+ = \{E\}$ \\
Try removing E$\{D\}^+ = \{D\}$ \\

Means the FD's is a minimal basis.
 \clearpage
\item What are the functional (and multivalue) dependencies that you expect to hold for each relation if any. If you don't expect any to hold, describe why not. \\
% Problem 1c goes below
R(A,B,C,D,E)\\
$B \rightarrow C$\\
$BE \rightarrow D$\\
$DC \rightarrow E$\\
$DE \rightarrow A$\\


For the above set of the functional dependencies the FD's that voilets are: \\ 
$B \rightarrow C$\\
$DC \rightarrow E$\\
$DE \rightarrow A$\\

now we must also consider all the functional dependencies that follows those listed above.\\

For FD $B \rightarrow C$ find closure of the FD's that follows\\
$\{AB\}^+ = \{ A,B,C\}$  --> this is a voilation\\
$\{BE\}^+ = \{ A,B,C,D,E\}$ \\
$\{BD\}^+ = \{ A,B,C,D,E\}$ \\

thus $AB \rightarrow C$ FD also voilates BCNF \\

For FD $DC \rightarrow E$ find closure of the FD's that follows\\
$\{ACD\}^+ = \{ A,D,C,E\}$  --> this is a voilation\\
$\{BCD\}^+ = \{ A,D,C,E,B\}$

thus $ACD \rightarrow E$ FD also voilates BCNF \\




For FD $DE \rightarrow A$ find closure of the FD's that follows\\
$\{CDE\}^+ = \{ A,D,C,E\}$  --> this is a voilation\\
$\{BCD\}^+ = \{ A,D,C,E,B\}$

thus $CDE \rightarrow A$ FD also voilates BCNF \\

All the functional dependencies with the voilations are:\\
$CDE \rightarrow A$\\
$ACD \rightarrow E$\\
$AB \rightarrow C$\\
$B \rightarrow C$\\
$DC \rightarrow E$\\
$DE \rightarrow A$\\
$CD \rightarrow A$\\
 \clearpage
\end{enumerate}

\section*{Part 2} 
Write a program to load the grade data into a PostgreSQL database called FakeUData that follows your schema.
You {\bf MUST} use the database called FakeUData, and should assume it will already be created for you without any tables or data in it.
You may {\bf NOT} hardcode usernames in your code, use the USER environmental variable instead if user is needed.
Your program can be written in C++ or python, you may {\bf NOT} use standalone SQL or text files that hold your queries. 
You may {\bf NOT} use shell calls to implement your program.
All your queries need to be in your code. 
If you choose to make a C++ program, you must include a makefile and call the program loadfakeu. 
Include a readme file with descriptions of any issues/problems. 
If you choose to make a python program you must specify which version of python you used, and must provide a loadfakeu bash script to launch your python program.
The loadfakeu program {\bf MUST} be able to take one optional argument (the directory where the CSV data files will be located). 
If the argument is omitted, the default is the current working directory.
Scripts that require greater than 10 minutes to load all of the data may lose points. 
\clearpage

\section*{Part 3}
Write another program to query your database to calculate the following values, put the results in your write up, some may be best described with a chart instead of raw values. 
Name your program queryfakeu, it must output the data values for the following queries. 
The query program does not have to do everything in the SQL queries, but should limit the amount of data transfered. 
For example it is acceptable to have one SQL query for each unit number (1 - 20) for 3a, but it would be unacceptable to pull all student data on a per student basis and calculate the results. 
\begin{enumerate}[label=\alph*.]
\item Calculate the percent of students that attempt 1 � 20 units of ABC or DEF per quarter for every unit increment (e.g. 1, 2, 3,�). \\
% Problem 2a goes below
\begin{center}
\begin{tabular}{|c|c|}
\hline
Units & Percentage of Students That Attempted Units \\ \hline
1     & 1.16\%                                      \\ \hline
2     & 1.5\%                                       \\ \hline
3     & 2.42\%                                      \\ \hline
4     & 10.54\%                                     \\ \hline
5     & 2.12\%                                      \\ \hline
6     & 1.31\%                                      \\ \hline
7     & 0.63\%                                      \\ \hline
8     & 2.25\%                                      \\ \hline
9     & 1.08\%                                      \\ \hline
10    & 0.59\%                                      \\ \hline
11    & 0.71\%                                      \\ \hline
12    & 3.65\%                                      \\ \hline
13    & 1.09\%                                      \\ \hline
14    & 0.28\%                                      \\ \hline
15    & 0.25\%                                      \\ \hline
16    & 0.47\%                                      \\ \hline
17    & 0.11\%                                      \\ \hline
18    & 0.34\%                                      \\ \hline
19    & 0.07\%                                      \\ \hline
20    & 0.4\%                                       \\ \hline
\end{tabular}
\end{center}
 \clearpage
\item Find the easiest and hardest instructors based upon the grades of all the students they have taught in their courses. Provide their name and the average grade they assigned. (Ignore P/NP, S/NS grades) \\
% Problem 2b goes below
% Problem 2b goes below



\begin{center}
\begin{tabular}{|c|c|c|}
\hline
Instructor marking scheme & Instructor Name  & AVG GPA \\ \hline
Easy     & Murphy, Melanie S.                             & 4        \\ \hline
Easy     & Powell, Liliana M.                           &4        \\ \hline
Easy     &  White, Sophia V.                        &4 \\ \hline
Easy     & Porter, Ryan E.                          &4    \\ \hline
Easy     & Thomas, Santiago G.                       &4      \\ \hline
Easy     & Mendoza, Anthony A.                        &4         \\ \hline
Easy     & Odonnell, Madison G.                     &4              \\ \hline
Easy     & Smith, Daniel N.                          &4       \\ \hline
Easy     & Bell, David Q.                          &4\\ \hline
Easy       &Calderon, Isabella A.                      &4             \\ \hline
Easy       & Norris, Nathan A.                          &4           \\ \hline
Easy       & Henderson, Alexa M.                      &4           \\ \hline
Easy       & Brooks, Michael Y.                        &4        \\ \hline
Easy       & Mckay, Isaac A.                            &4    \\ \hline
Easy       & Gordon, Noah R.                          &4         \\ \hline
Easy       & Hart, Vincent B.                          &4       \\ \hline
Easy       & Mcclure, Noah P.                           &4                                 \\ \hline
Easy       & Hayes, Aaliyah D.                          &4          \\ \hline
Easy       & Allen, Mateo G.                          &4      \\ \hline
Easy       & Taylor, Liam J.                            &4      \\ \hline
Easy       &Turner, Emily A.                              &4     \\ \hline
Easy       & Fletcher, Arianna J.                           &4          \\ \hline
Easy       & Russo, Angel J.                                &4\\ \hline
Easy       & Morris, Ariana O.                            &4    \\ \hline
Easy       & Walker, Evelyn K.                             &4   \\ \hline
Easy       & Jackson, Dominic E.                            &4     \\ \hline

Difficult       & Houston, Noah J.                         &0        \\ \hline
Difficult       & Morris, Wyatt J.                          &0       \\ \hline
Difficult       & Williamson, Jasmine B.                          &0          \\ \hline
Difficult       & Love, Jeremiah A.                       &0      \\ \hline

\end{tabular}
\end{center}






 \clearpage
\item Calculate the average GPA for the students that take each number of units from part a. Assume that the grades have standard grade points (A+ = 4.0, A = 4.0, A- = 3.7, B+ = 3.3�). \\
% Problem 2c goes below
\begin{center}
\begin{tabular}{|c|c|}
\hline
Units & Average GPA of Students That Attempted Units \\ \hline
1     & 3.67\                                      \\ \hline
2     & 3.58\                                       \\ \hline
3     & 3.21\                                      \\ \hline
4     & 3.12\                                     \\ \hline
5     & 3.23\                                      \\ \hline
6     & 3.63\                                      \\ \hline
7     & 3.73\                                      \\ \hline
8     & 3.26\                                      \\ \hline
9     & 3.47\                                      \\ \hline
10    & 3.19\                                      \\ \hline
11    & 4.0\                                      \\ \hline
12    & 3.46\                                      \\ \hline
13    & 3.0\                                      \\ \hline
14    & 3.75\                                      \\ \hline
15    & 3.44\                                      \\ \hline
16    & 3.57\                                      \\ \hline
17    & 3.98\                                      \\ \hline
18    & 3.80\                                      \\ \hline
19    & 3.0\                                      \\ \hline
20    & 3.57\                                       \\ \hline
\end{tabular}
\end{center} \clearpage
\item Find the courses with the highest and lowest pass rates. Assume that F, NP, and NS are not passing grades. \\
% Problem 2d goes below
Courses with highest passing rate: \\ \\
ABC364	DEF359	ABC250	DEF423	ABC349	DEF241	DEF363	DEF404	DEF329	ABC325	
ABC246	ABC372	ABC353	DEF263	DEF247	DEF365	ABC368	DEF381	ABC333	DEF322	
ABC206	DEF272	DEF360	DEF210	DEF233	DEF358	ABC322	DEF284	ABC243	DEF338
DEF380	DEF321	ABC302	DEF266	DEF400	ABC311	ABC228	ABC304	DEF410
DEF309
ABC366
DEF399
ABC309
DEF394
DEF343
DEF361
DEF260
DEF302
DEF346
DEF282
DEF334
DEF418
DEF419
ABC323
ABC359
ABC313
ABC233
ABC365
DEF420
ABC310
DEF328
DEF264
DEF372
DEF265
ABC341
DEF421
ABC318
DEF339
DEF283
ABC208
ABC231
ABC247
ABC316
DEF296
DEF295
ABC324
ABC377
DEF267
DEF350
ABC301
ABC336
ABC225
DEF274
DEF268
DEF387
ABC362
ABC212
DEF362
DEF407
DEF224
ABC203
DEF262
DEF390
DEF231
DEF401
ABC348
ABC314
ABC335
ABC219
ABC369
ABC352
ABC235
ABC356
DEF287
ABC102
DEF271
ABC205
DEF345
DEF246
DEF383
DEF333
ABC340
DEF376
DEF388
ABC237
DEF385
DEF353
DEF316
DEF374
DEF319
DEF212
DEF317
DEF405
DEF351
DEF375
ABC258
DEF217
DEF280
DEF223
ABC331
DEF281
ABC338
DEF311
DEF297
ABC343
ABC350
DEF327
ABC307
DEF232
DEF398
ABC346
DEF213
ABC373
ABC320
DEF395
DEF379
DEF270
DEF228
ABC332
ABC249
DEF275
DEF289
ABC240
DEF303
DEF352
ABC259
DEF337
DEF222
DEF425
DEF424
DEF340
DEF291
ABC306
DEF306
ABC317
ABC260
ABC354
DEF326
ABC224
ABC315
ABC367
DEF313
DEF240
DEF202
DEF344
ABC110
DEF335
DEF314
ABC360
DEF105
ABC339
DEF368
DEF218
DEF397
DEF102
ABC337
DEF277
DEF273
DEF402
DEF354
ABC312
ABC363
DEF227
DEF382
ABC326
ABC351
DEF301
ABC355
ABC344
DEF288
DEF377
DEF290
DEF237
DEF349
DEF366
ABC305
ABC328
DEF325
ABC345
ABC347
ABC255
ABC204
DEF336
DEF225
DEF226
ABC236
ABC303
ABC319
DEF371
DEF236
ABC215
DEF348
DEF323
DEF278
DEF315
ABC308
DEF261
DEF408
ABC207
ABC361
ABC334
ABC223
DEF312
DEF364
DEF370
ABC374
DEF310
DEF384
DEF276
DEF392
DEF427
DEF305
ABC242
ABC342
DEF378
DEF320
DEF386
DEF357
DEF298
DEF244
DEF396
ABC357
DEF269
ABC109
DEF324
DEF216
ABC327
DEF355
DEF331
DEF342
DEF403
ABC254
DEF422
ABC218\\ \\
Courses with lowest passing rate: \\ \\
DEF307
DEF207
ABC214
DEF341
 \clearpage
\item Find the list of courses that must be cross listed as they have the same meeting times during the normal quarters. Only list the pair once, put the course name/number string in alphabetically order of the pairs. \\
% Problem 2e goes below

\begin{center}
\begin{tabular}{|c|}
\hline
LIST OF COURSES THAT ARE CROSSLISTED  \\ \hline
106                                        \\ \hline
110                                        \\ \hline
112                                      \\ \hline
113                                   \\ \hline
225                                       \\ \hline


245                                        \\ \hline
251                                        \\ \hline
253                                      \\ \hline
254                                   \\ \hline
256                                       \\ \hline



261                                        \\ \hline
285                                        \\ \hline
286                                      \\ \hline
287                                   \\ \hline
300                                       \\ \hline


371                                      \\ \hline
376                                        \\ \hline
378                                      \\ \hline
379                                   \\ \hline
416                                       \\ \hline
426 \\ \hline

\end{tabular}
\end{center}
 \clearpage
\item Find the major that performs the best/worst on average in ABC courses. Repeat the analysis for DEF courses as well. \\
% Problem 2f goes below
% Problem 2f goes below
ABC2,
O239,
OT31,
OT99,
OT26,
O223,
OTH1,
OT30,
O251,
O194
,OT39
,OTH6,
OTH8,
O238,
OTH3
,OTH9,
O257,
OTH4,
OT48,
O155,
O218,
OT71,
OT54,
OT82,
O143,
O130,
O192,
O225,
OT40,
O224,
OT14,
OT93,
O174,
O110,
O189,
OT11,
O187,
ABC1,
OT67,
O241
,OT94,
OT83,
O248,
O198,
,O214,
OT55,
O118,
OT17,
O108,
OT28,
O227,
OT22,
OT59,
O112,
O127,
O200,
OT73,
O123
,OT77,
O183,
DEF2,
O240,
O190,
DEF1,
O278,
OT16,
OT44,
O252,
OT69,
OTH2,
ABCG,
O101,
O244,
OT21,
OT80,
O135,
O216,
OT72,
O160,
O188,
OT42,
O284,
O262,
OT68,
OTH5,
OT76,
O107,
OT20,
OT60,
O146,
O212,
O164,
OT98,
OT25,
OT27,
OT34,
OTH7,
O181,
O159,
O115,
OT75,
OT35,
OT12,
O204,
OT37,
O103,
OT64,
O161,
O268,
O131,
O250,
O253,
O116,
O104,
DEFG,
O184,
OT15,
OT62,
O243,
OT13,
O237,
OT74,
O254,
OT10,
OT78,
OT56,
OT41,
O117
 \clearpage
\item Find the top 5 majors that students transfer from into ABC. What is the percent of students from each of those majors compared to overall transfers? \\
% Problem 2g goes below
% Problem 2g goes below
. \\ \\
% Problem 2a goes below
\begin{center}
\begin{tabular}{|c|c|}
\hline
Major & Percentage of Students transfer from to ABC \\ \hline
DEF2    & 74.5629140909743\%                                      \\ \hline
OT35    & 8.00156442662663\%                                       \\ \hline
DEF1     & 5.59421956659578\%                                      \\ \hline
DEFG   & 4.9412700778369\%                                     \\ \hline
OTH7     &2.26566613081383\%                                      \\ \hline

\end{tabular}
\end{center}
 \clearpage
\item Find the top 5 majors that students transfer to from ABC. What is the percent of students to each of those majors compared to overall transfers out? \\
% Problem 2h goes below
% Problem 2h goes below
% Problem 2g goes below
. \\ \\

% Problem 2a goes below
\begin{center}
\begin{tabular}{|c|c|}
\hline
Major & Percentage of Students transfer from  ABC to other major \\ \hline
DEF2    & 74.5629140909743\%                                      \\ \hline
OT35    & 8.00156442662663\%                                       \\ \hline
DEF1     & 5.59421956659578\%                                      \\ \hline
DEFG   & 4.9412700778369\%                                     \\ \hline
OTH7     &2.26566613081383\%                                      \\ \hline

\end{tabular}
\end{center}
 \clearpage
\end{enumerate}

\section*{Part 4}
Extra credit: The Efficient XML Interchange (EXI) is a format for the compact representation of XML information. 
The CSV files provided for this assignment have been consolidated into a single EXI file (HW4Grades.exi) that is available in the resources section of Canvas. 
Implement a separate program that it can load the database from the EXI file. 
You may {\bf NOT} use shell calls, or creation of external temporary files for this part.
Name your program or bash script loadfakeuexi.
\clearpage

\section*{Part 5}
Extra credit: Additional queries/query program.
\begin{enumerate}[label=\alph*.]
\item Find the courses that appear to be prerequisites for ABC 203, ABC 210, and ABC 222. For this problem list the courses that the X\% of students have taken for every 5\% increment from 50\% - 100\% prior to taking the course. (Add this output to your query program.)\\
% Problem 2a goes below
 \clearpage
\item Write a program that will find an open room for course expansion. The program must prompt for term, CID, and number students to add. The room(s) returned should be ordered from best to worst fit with up to 5 results. Assume that each room capacity is the maximum number of students listed for any particular meeting in the data files (don't forget that lectures may be split across multiple CIDs). Name this program findroomfakeu.
% Problem 2b goes below
 \clearpage
\end{enumerate}



\end{document}